\PassOptionsToPackage{unicode=true}{hyperref} % options for packages loaded elsewhere
\PassOptionsToPackage{hyphens}{url}
%
\documentclass[
  ignorenonframetext,
]{beamer}
\usepackage{pgfpages}
\setbeamertemplate{caption}[numbered]
\setbeamertemplate{caption label separator}{: }
\setbeamercolor{caption name}{fg=normal text.fg}
\beamertemplatenavigationsymbolsempty
% Prevent slide breaks in the middle of a paragraph:
\widowpenalties 1 10000
\raggedbottom
\setbeamertemplate{part page}{
  \centering
  \begin{beamercolorbox}[sep=16pt,center]{part title}
    \usebeamerfont{part title}\insertpart\par
  \end{beamercolorbox}
}
\setbeamertemplate{section page}{
  \centering
  \begin{beamercolorbox}[sep=12pt,center]{part title}
    \usebeamerfont{section title}\insertsection\par
  \end{beamercolorbox}
}
\setbeamertemplate{subsection page}{
  \centering
  \begin{beamercolorbox}[sep=8pt,center]{part title}
    \usebeamerfont{subsection title}\insertsubsection\par
  \end{beamercolorbox}
}
\AtBeginPart{
  \frame{\partpage}
}
\AtBeginSection{
  \ifbibliography
  \else
    \frame{\sectionpage}
  \fi
}
\AtBeginSubsection{
  \frame{\subsectionpage}
}
\usepackage{lmodern}
\usepackage{amssymb,amsmath}
\usepackage{ifxetex,ifluatex}
\ifnum 0\ifxetex 1\fi\ifluatex 1\fi=0 % if pdftex
  \usepackage[T1]{fontenc}
  \usepackage[utf8]{inputenc}
  \usepackage{textcomp} % provides euro and other symbols
\else % if luatex or xelatex
  \usepackage{unicode-math}
  \defaultfontfeatures{Scale=MatchLowercase}
  \defaultfontfeatures[\rmfamily]{Ligatures=TeX,Scale=1}
\fi
% use upquote if available, for straight quotes in verbatim environments
\IfFileExists{upquote.sty}{\usepackage{upquote}}{}
\IfFileExists{microtype.sty}{% use microtype if available
  \usepackage[]{microtype}
  \UseMicrotypeSet[protrusion]{basicmath} % disable protrusion for tt fonts
}{}
\makeatletter
\@ifundefined{KOMAClassName}{% if non-KOMA class
  \IfFileExists{parskip.sty}{%
    \usepackage{parskip}
  }{% else
    \setlength{\parindent}{0pt}
    \setlength{\parskip}{6pt plus 2pt minus 1pt}}
}{% if KOMA class
  \KOMAoptions{parskip=half}}
\makeatother
\usepackage{xcolor}
\IfFileExists{xurl.sty}{\usepackage{xurl}}{} % add URL line breaks if available
\IfFileExists{bookmark.sty}{\usepackage{bookmark}}{\usepackage{hyperref}}
\hypersetup{
  pdftitle={Introduction to statistical inference},
  pdfauthor={Brian Caffo, Jeff Leek, Roger Peng},
  pdfborder={0 0 0},
  breaklinks=true}
\urlstyle{same}  % don't use monospace font for urls
\newif\ifbibliography
\usepackage{graphicx,grffile}
\makeatletter
\def\maxwidth{\ifdim\Gin@nat@width>\linewidth\linewidth\else\Gin@nat@width\fi}
\def\maxheight{\ifdim\Gin@nat@height>\textheight\textheight\else\Gin@nat@height\fi}
\makeatother
% Scale images if necessary, so that they will not overflow the page
% margins by default, and it is still possible to overwrite the defaults
% using explicit options in \includegraphics[width, height, ...]{}
\setkeys{Gin}{width=\maxwidth,height=\maxheight,keepaspectratio}
\setlength{\emergencystretch}{3em}  % prevent overfull lines
\providecommand{\tightlist}{%
  \setlength{\itemsep}{0pt}\setlength{\parskip}{0pt}}
\setcounter{secnumdepth}{-2}

% set default figure placement to htbp
\makeatletter
\def\fps@figure{htbp}
\makeatother


\title{Introduction to statistical inference}
\subtitle{Statistical inference}
\author{Brian Caffo, Jeff Leek, Roger Peng}
\date{}
\logo{\includegraphics{bloomberg\_shield.png}}

\begin{document}
\frame{\titlepage}

\begin{frame}
  \tableofcontents[hideallsubsections]
\end{frame}
\begin{frame}{Statistical inference defined}
\protect\hypertarget{statistical-inference-defined}{}

Statistical inference is the process of drawing formal conclusions from
data.

In our class, we wil define formal statistical inference as settings
where one wants to infer facts about a population using noisy
statistical data where uncertainty must be accounted for.

\end{frame}

\begin{frame}{Motivating example: who's going to win the election?}
\protect\hypertarget{motivating-example-whos-going-to-win-the-election}{}

In every major election, pollsters would like to know, ahead of the
actual election, who's going to win. Here, the target of estimation (the
estimand) is clear, the percentage of people in a particular group
(city, state, county, country or other electoral grouping) who will vote
for each candidate.

We can not poll everyone. Even if we could, some polled may change their
vote by the time the election occurs. How do we collect a reasonable
subset of data and quantify the uncertainty in the process to produce a
good guess at who will win?

\end{frame}

\begin{frame}{Motivating example: is hormone replacement therapy
effective?}
\protect\hypertarget{motivating-example-is-hormone-replacement-therapy-effective}{}

A large clinical trial (the Women's Health Initiative) published results
in 2002 that contradicted prior evidence on the efficacy of hormone
replacement therapy for post menopausal women and suggested a negative
impact of HRT for several key health outcomes. \textbf{Based on a
statistically based protocol, the study was stopped early due an excess
number of negative events.}

Here's there's two inferential problems.

\begin{enumerate}
\tightlist
\item
  Is HRT effective?
\item
  How long should we continue the trial in the presence of contrary
  evidence?
\end{enumerate}

See WHI writing group paper JAMA 2002, Vol 288:321 - 333. for the paper
and Steinkellner et al.~Menopause 2012, Vol 19:616 621 for adiscussion
of the long term impacts

\end{frame}

\begin{frame}{Motivating example}
\protect\hypertarget{motivating-example}{}

\begin{block}{Brain activation}

\begin{figure}
\centering
\includegraphics{fig/fmri-salmon.jpg}
\caption{fMRI salmon study}
\end{figure}

\url{http://www.wired.com/2009/09/fmrisalmon/}

\end{block}

\end{frame}

\begin{frame}{Summary}
\protect\hypertarget{summary}{}

\begin{itemize}
\tightlist
\item
  These examples illustrate many of the difficulties of trying to use
  data to create general conclusions about a population.
\item
  Paramount among our concerns are:

  \begin{itemize}
  \tightlist
  \item
    Is the sample representative of the population that we'd like to
    draw inferences about?
  \item
    Are there known and observed, known and unobserved or unknown and
    unobserved variables that contaminate our conclusions?
  \item
    Is there systematic bias created by missing data or the design or
    conduct of the study?
  \item
    What randomness exists in the data and how do we use or adjust for
    it? Here randomness can either be explicit via randomization or
    random sampling, or implicit as the aggregation of many complex
    uknown processes.
  \item
    Are we trying to estimate an underlying mechanistic model of
    phenomena under study?
  \end{itemize}
\item
  Statistical inference requires navigating the set of assumptions and
  tools and subsequently thinking about how to draw conclusions from
  data.
\end{itemize}

\end{frame}

\begin{frame}{Example tools of the trade}
\protect\hypertarget{example-tools-of-the-trade}{}

\begin{enumerate}
\tightlist
\item
  Randomization: concerned with balancing unobserved variables that may
  confound inferences of interest
\item
  Random sampling: concerned with obtaining data that is representative
  of the population of interest
\item
  Sampling models: concerned with creating a model for the sampling
  process, the most common is so called ``iid''.
\item
  Hypothesis testing: concerned with decision making in the presence of
  uncertainty
\item
  Confidence intervals: concerned with quantifying uncertainty in
  estimation
\item
  Probability models: a formal connection between the data and a
  population of interest. Often probability models are assumed or are
  approximated.
\item
  Study design: the process of designing an experiment to minimize
  biases and variability.
\item
  Nonparametric bootstrapping: the process of using the data to, with
  minimal probability model assumptions, create inferences.
\item
  Permutation, randomization and exchangeability testing: the process of
  using data permutations to perform inferences.
\end{enumerate}

\end{frame}

\begin{frame}{In this class}
\protect\hypertarget{in-this-class}{}

\begin{itemize}
\tightlist
\item
  In this class, we will primarily focus on basic sampling models, basic
  probability models and frequency style analyses to create standard
  inferences.
\item
  Being data scientists, we will also consider some inferential
  strategies that rely heavily on the observed data, such as permutation
  testing and bootstrapping.
\item
  As probability modeling will be our starting point, we first build up
  basic probability.
\end{itemize}

\end{frame}

\begin{frame}{Where to learn more on the topics not covered}
\protect\hypertarget{where-to-learn-more-on-the-topics-not-covered}{}

\begin{enumerate}
\tightlist
\item
  Explicit use of random sampling in inferences: look in references on
  ``finite population statistics''. Used heavily in polling and sample
  surveys.
\item
  Explicit use of randomization in inferences: look in references on
  ``causal inference'' especially in clinical trials.
\item
  Bayesian probability and Bayesian statistics: look for basic
  itroductory books (there are many).
\item
  Missing data: well covered in biostatistics and econometric
  references; look for references to ``multiple imputation'', a popular
  tool for addressing missing data.
\item
  Study design: consider looking in the subject matter area that you are
  interested in; some examples with rich histories in design:
\item
  The epidemiological literature is very focused on using study design
  to investigate public health.
\item
  The classical development of study design in agriculture broadly
  covers design and design principles.
\item
  The industrial quality control literature covers design thoroughly.
\end{enumerate}

\end{frame}

\end{document}
