\documentclass[]{article}
\usepackage{geometry}
 \geometry{
 a4paper,
 total={170mm,257mm},
 left=20mm,
 top=20mm,
 }
\usepackage{graphicx}
\usepackage{subcaption}

\usepackage[justification=centering]{caption}
\usepackage[hidelinks]{hyperref}
\usepackage{enumitem}
\usepackage[utf8]{inputenc}
\usepackage{float}
\DeclareTextFontCommand{\helvetica}{\fontfamily{phv}\selectfont}
\setlength{\parindent}{4em}
\setlength{\parskip}{1em}
\linespread{1.5}

\usepackage[table]{xcolor}



\title{PAC2  Desenvolupament el treball - Fase 1}
\date{18 de Març 2019}
\author{Vasyl Druchkiv \\ Estudiant de Màster de Bioestadística i Bioinformàtica}
\renewcommand{\contentsname}{Índice}
\usepackage{setspace}

\begin{document}
\maketitle
\makeatletter
\renewcommand{\@seccntformat}[1]{}
\makeatother
\begin{spacing}{0.1}
\tableofcontents
\end{spacing}

L'estructura recomanada per realitzar l'informe de seguiment és la següent:

    \section{Identificació del treball i data de l'informe}

    \section{Descripció de l'avanç del projecte} 

     \subsection{Grau de compliment dels objectius i resultats previstos en el pla de treball.}

     \subsection{Justificació dels canvis en cas necessari}

   \section{Relació de les activitats realitzades}

         \subsection{Activitats previstes en el pla de treball}

         \subsection{Activitats no previstes i realitzades o programes}

   \section{Relació de les desviacions en la temporització i accions de mitigació si escau i actualització del cronograma si escau}

   \section{Llistat dels resultats parcials obtinguts fins al moment (lliurables que s'adjunten)}

   \section{Comentaris del vostre director particular si ho considereu necessari.}

\end{document}





































